\documentclass[a4paper,11pt]{article}
\usepackage{latexsym}
\usepackage[utf8]{inputenc}
\usepackage[activeacute,spanish]{babel}
\usepackage{lineno}
\usepackage{amsmath}
\usepackage{graphicx}
\usepackage{booktabs}

\usepackage[colorlinks,linkcolor={blue},citecolor={blue},urlcolor={red}]{hyperref}
\hypersetup{urlcolor=blue, colorlinks=true} % Colors hyperlinks in blue

\linespread{1.2} % Interlineado
\usepackage{geometry} % Control de margenes
 \geometry{
 left=22mm,
 right=22mm,
 top=22mm,
 bottom=22mm
 }

%opening
\title{Formato Básico de Artículo de Investigación}
\author{César Jiménez\\
 Universidad Nacional Mayor de San Marcos \\
\emph{cjimenezt@unmsm.edu.pe}\\}

\begin{document}
\linenumbers{} % Coloca numeracion a las lineas
\maketitle{}

\begin{abstract} \noindent{}
	Aqui va el Resumen del trabajo de investigación. Este formato en Latex puede ser utilizado durante el proceso de revisión por pares.\\
	\textbf{Palabras clave}: clave1, clave2, clave3.
\end{abstract}

\centerline{\textbf{Basic Format of Research Paper}}
\renewcommand{\abstractname}{Abstract}

\begin{abstract} \noindent{}
	Here is the Abstract of the research. This format in Latex can be used in the peer review process.\\
	\textbf{Keywords}: key1, key2, key3.
\end{abstract}


\section{Introducción}
Esto es la introduccion. Las citaciones se realizan de esta manera: \cite{1}. Las referencias se colocan de acuerdo al orden de aparición de las citaciones \cite{2}, de acuerdo al formato APS \cite{3}.

\begin{equation}
	F_e = k \frac{q_1 q_{2}}{r^2}
\end{equation}

\subsection*{Área de estudio}

\subsection*{Antecedentes}

\section{Datos}
Ejemplo de ecuacion:
\begin{equation}
	\label{1}
	\ {s (t)} = u (t) * f (Q,t) * I (t) \
\end{equation}

\section{Metodología}
Ejemplo de ecuacion larga:
\begin{align}
	\frac{\partial{} M}{\partial{} t} + \frac{\partial}{\partial{} x} \left(\frac{M^2}{D}\right) + \frac{\partial}{\partial{} x} \left(\frac{MN}{D}\right) & = -gD\frac{\partial\eta}{\partial{} x} \nonumber{} \\
	                                                                                                                                                       & \quad{} -\frac{gn^2}{D^{7/3}}M\sqrt{M^2+N^2}
	\label{2}
\end{align}

% Ejemplo de Figura
\begin{figure}
	\centerline{\includegraphics[scale=0.64]{images/figura.eps}}
	\caption{Ejemplo de Figura (en formato eps Encapsulated Post Script).}
	\label{figura1}
\end{figure}

% Ejemplo de Tabla
\begin{table}
	\centerline{
		\begin{tabular}{l c c c r}
			\toprule{}
			\(N\) & \(v_p (km/s)\) & \(v_s (km/s)\) & \(\rho (g/cm^3)\) & \(t (km)\) \\
			\midrule{}
			1     & 1.50           & 0.00           & 1.02              & 4.2        \\
			2     & 5.66           & 3.23           & 2.60              & 2.0        \\
			3     & 5.92           & 3.38           & 2.60              & 8.0        \\
			4     & 6.20           & 3.54           & 2.90              & 12.0       \\
			5     & 6.44           & 3.68           & 3.38              & 8.0        \\
			6     & 6.87           & 3.92           & 3.38              & 20.0       \\
			7     & 7.92           & 4.52           & 3.37              & 0.0        \\
			\bottomrule{}
		\end{tabular}}
	\smallskip{}
	\caption{Modelo de Tabla.}
	\label{tabla1}
\end{table}

\section{Resultados y Discusión}
Ecuacion con una integral:
\begin{equation}
	\label{eq:3}
	M_0=\frac{4\pi\rho{} v^3 R_t}{R_{\theta{} \varphi}}\int_{\tau_1}^{\tau_2} s (t)\,dt \
\end{equation}

\begin{equation}
	\label{eq:4}
	A=\int_{a}^{b} f (x)\,dx \
\end{equation}

La ecuación de Schrodinger \( \hat{H} \Psi = E \Psi \) es una ecuación de valores propios.

\begin{equation}
	i\hbar\frac{\partial}{\partial{} t}\left|\Psi(t)\right>=H\left|\Psi(t)\right>
\end{equation}

\section{Conclusiones}
Aqui van las conclusiones

De acuerdo a la Figura \ref{figura1}, los valores maximo y minimo son 2.5 y -0.8.

\section*{Agradecimientos}
Aqui van los agradecimientos. Primero se agradece a las personas y luego a las instituciones.

\begin{thebibliography}{99}
	\bibitem{1} N. Apellido. Titulo del artículo. Rev. Inv. Fis., \textbf{21} (1), 18\textendash26 (2018).

	\bibitem{2} H. Benny y J. Pérez. \emph{Título de Libro}. Editorial San Marcos, Lima (2018).

	\bibitem{3} Y. Okada. Internal deformation in a half space. Bull. Seismol. Soc.Am. \textbf{82} (2) 1018\textendash1040 (1992).

	\bibitem{4} M. Bazant and J. Bush. Beyond six feet: A guideline to limit indoor airborne transmission of COVID-19, doi: \url{http://doi.org/10.1101/2020.08.26.20182824}.

	\bibitem{5} M. Kikuchi y H. Kanamori. \emph{Notes on Teleseismic Body-Wave Inversion Program} \url{http://www.eri.u-tokyo.ac.jp/ETAL/KIKUCHI} (2003). % chktex 8

	\bibitem{6} H. Pulker. \emph{Coatings On Glass}. Elsevier Science, 2nd edition. Amsterdam (1999).

\end{thebibliography}

\end{document}
